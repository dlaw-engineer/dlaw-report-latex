%%%%%% Run at command line, run
%%%%%% xelatex grad-sample.tex 
%%%%%% for a few times to generate the output pdf file
\documentclass[12pt,oneside,openright,a4paper]{cpe-thai-project}


\usepackage{polyglossia}
\usepackage{titlesec}
\setdefaultlanguage{thai}
\setotherlanguage{english}
\newfontfamily\thaifont[Script=Thai,Scale=1.23]{TH Sarabun New}
\defaultfontfeatures{Mapping=tex-text,Scale=1.23,LetterSpace=0.0}
\setmainfont[Scale=1.23,LetterSpace=0,WordSpace=1.0,FakeStretch=1.0,Mapping=tex-text]{TH Sarabun New}
\XeTeXlinebreaklocale "th"	
\XeTeXlinebreakskip = 0pt plus 0pt
\emergencystretch=10pt


%%%%%%%%%%%%%%%%%%%%%%%%%%%%%%%%%%%%%%%%%%%%%%%%%%%%%%%%%%%%%%%%%%%
% Customize below to suit your needs 
% The ones that are optional can be left blank. 
%%%%%%%%%%%%%%%%%%%%%%%%%%%%%%%%%%%%%%%%%%%%%%%%%%%%%%%%%%%%%%%%%%%
% First line of title
\def\disstitleone{Document management system for lawyer}   
% Second line of title
% \def\disstitletwo{Project/Indep title line 2 (optional)}   
% Your first name and lastname
\def\dissauthor{Mr. Kittipat Dechkul}   % 1st member
\def\dissauthortwo{Mr. Choolerk Taebanpakul}   % 2nd member (optional)
\def\dissauthorthree{}   % 2nd member (optional)


% The degree that you're persuing..
\def\dissdegree{Bachelor of Engineering} % Name of the degree
\def\dissdegreeabrev{B.Eng} % Abbreviation of the degree
\def\dissyear{2022}                   % Year of submission
\def\thaidissyear{2565}               % Year of submission (B.E.)

%%%%%%%%%%%%%%%%%%%%%%%%%%%%%%%%%%%%%%%%%%%%
% Your project and independent study committee..
%%%%%%%%%%%%%%%%%%%%%%%%%%%%%%%%%%%%%%%%%%%%
\def\dissadvisor{Assoc.Prof. Priyakorn Pusawiro , Ph.D.}  % Advisor
\def\disscoadvisor{}  % Co-advisor
\def\disscommitteetwo{Assoc.Prof. Sanan Srakaew, Ph.D.}  % 3rd committee member (optional)
\def\disscommitteethree{Asst.Prof. Suthathip Maneewongpathana, Ph.D.}   % 4th committee member (optional) 
\def\disscommitteefour{Mr. Kharittha Jangsamsee, Ph.D.}    % 5th committee member (optional) 

\def\worktype{Project} %%  Project or Independent study
\def\disscredit{3}   %% 3 credits or 6 credits


\def\fieldofstudy{Computer Engineering} 
\def\department{Computer Engineering} 
\def\faculty{Engineering}

\def\thaifieldofstudy{วิศวกรรมคอมพิวเตอร์} 
\def\thaidepartment{วิศวกรรมคอมพิวเตอร์} 
\def\thaifaculty{วิศวกรรมศาสตร์}
 
\def\appendixnames{Appendix} %%% Appendices or Appendix

\def\thaiworktype{ปริญญานิพนธ์} %  Project or research project % 
\def\thaidisstitleone{Document management system for lawyer}
% \def\thaidisstitletwo{หัวข้อปริญญานิพนธ์บรรทัดสอง}
\def\thaidissauthor{นายกิตติภัฎ เดชกุล}
\def\thaidissauthortwo{นายชูฤกษ์ แต่บรรพกุล} %Optional
\def\thaidissauthorthree{} %Optional

\def\thaidissadvisor{รศ.ดร. ปริยกร ปุสวิโร}
%% Leave this empty if you have no co-advisor
% \def\thaidisscoadvisor{นายสุรสิทธิ์ ประคุณหังสิต} %Optional
\def\thaidissdegree{วิศวกรรมศาสตรบัณฑิต}

% Change the line spacing here...
\linespread{1.15}

%%%%%%%%%%%%%%%%%%%%%%%%%%%%%%%%%%%%%%%%%%%%%%%%%%%%%%%%%%%%%%%%
% End of personal customization.  Do not modify from this part 
% to \begin{document} unless you know what you are doing...
%%%%%%%%%%%%%%%%%%%%%%%%%%%%%%%%%%%%%%%%%%%%%%%%%%%%%%%%%%%%%%%%


%%%%%%%%%%%% Dissertation style %%%%%%%%%%%
%\linespread{1.6} % Double-spaced  
%%\oddsidemargin    0.5in
%%\evensidemargin   0.5in
%%%%%%%%%%%%%%%%%%%%%%%%%%%%%%%%%%%%%%%%%%%
%\renewcommand{\subfigtopskip}{10pt}
%\renewcommand{\subfigbottomskip}{-5pt} 
%\renewcommand{\subfigcapskip}{-6pt} %vertical space between caption
%                                    %and figure.
%\renewcommand{\subfigcapmargin}{0pt}

\renewcommand{\topfraction}{0.85}
\renewcommand{\textfraction}{0.1}

\newtheorem{theorem}{Theorem}
\newtheorem{lemma}{Lemma}
\newtheorem{corollary}{Corollary}

\def\QED{\mbox{\rule[0pt]{1.5ex}{1.5ex}}}
\def\proof{\noindent\hspace{2em}{\itshape Proof: }}
\def\endproof{\hspace*{\fill}~\QED\par\endtrivlist\unskip}
%\newenvironment{proof}{{\sc Proof:}}{~\hfill \blacksquare}
%% The hyperref package redefines the \appendix. This one 
%% is from the dissertation.cls
%\def\appendix#1{\iffirstappendix \appendixcover \firstappendixfalse \fi \chapter{#1}}
%\renewcommand{\arraystretch}{0.8}
%%%%%%%%%%%%%%%%%%%%%%%%%%%%%%%%%%%%%%%%%%%%%%%%%%%%%%%%%%%%%%%%
%%%%%%%%%%%%%%%%%%%%%%%%%%%%%%%%%%%%%%%%%%%%%%%%%%%%%%%%%%%%%%%%

\usepackage{ragged2e}
\begin{document}

\pdfstringdefDisableCommands{%
\let\MakeUppercase\relax
}

\begin{center}
  \includegraphics[width=2.8cm]{logo02.jpg}
\end{center}
\vspace*{-1cm}

\maketitlepage
\makesignaturepage 

%%%%%%%%%%%%%%%%%%%%%%%%%%%%%%%%%%%%%%%%%%%%%%%%%%%%%%%%%%%%%%
%%%%%%%%%%%%%%%%%%%%%% English abstract %%%%%%%%%%%%%%%%%%%%%%%
%%%%%%%%%%%%%%%%%%%%%%%%%%%%%%%%%%%%%%%%%%%%%%%%%%%%%%%%%%%%%%
\abstract

In a multihop ad hoc network, the interference among nodes is
  reduced to maximize the throughput by using a smallest transmission
  range that still preserve the network connectivity. However, most
  existing works on transmission range control focus on the
  connectivity but lack of results on the throughput performance. This
  paper analyzes the per-node saturated throughput of an IEEE 802.11b
  multihop ad hoc network with a uniform transmission range. Compared
  to simulation, our model can accurately predict the per-node
  throughput.  The results show that the maximum achievable per-node
  throughput can be as low as 11\% of the channel capacity in a normal
  set of $\alpha$ operating parameters independent of node density. However, if
  the network connectivity is considered, the obtainable throughput
  will reduce by as many as 43\% of the maximum throughput. 

\begin{flushleft}
\begin{tabular*}{\textwidth}{@{}lp{0.8\textwidth}}
\textbf{Keywords}: & Web application / (DMS) Document Management System / (OCR) Optical Character Recognition
\end{tabular*}
\end{flushleft}
\endabstract

%%%%%%%%%%%%%%%%%%%%%%%%%%%%%%%%%%%%%%%%%%%%%%%%%%%%%%%%%%%%%%
%%%%%%%%%% Thai abstract here %%%%%%%%%%%%%%%%%%%%%%%%%%%%%%%%%
%%%%%%%%%%%%%%%%%%%%%%%%%%%%%%%%%%%%%%%%%%%%%%%%%%%%%%%%%%%%%%
% {\newfontfamily\thaifont{TH Sarabun New:script=thai}[Scale=1.3]
% \XeTeXlinebreaklocale "th_TH"	
% \thaifont
\thaiabstract

เซ็นเซอร์ เอ็กซ์เพรสรองรับคอนเซปต์สหัสวรรษเมจิก อิ่มแปร้ เฟรชชี่ ชาร์ปเช็งเม้งคลาสสิก แพตเทิร์น แอลมอนด์ เพลซว้อยก๊วน ซาร์ดีนซี้เนิร์สเซอรีอีสต์ สเตเดียมเพียบแปร้โอ้ยแคมปัส จัมโบ้ช็อตแมคเคอเรลอึ๋ม สตริง แมกกาซีนสตริงผ้าห่ม ฮัลโหล ยิม รอยัลตี้

\begin{flushleft}
\begin{tabular*}{\textwidth}{@{}lp{0.8\textwidth}}
 & \\

\textbf{คำสำคัญ}: & การชุบเคลือบด้วยไฟฟ้า / การชุบเคลือบผิวเหล็ก /  เคลือบผิวรังสี
\end{tabular*}
\end{flushleft}
\endabstract

%}

%%%%%%%%%%%%%%%%%%%%%%%%%%%%%%%%%%%%%%%%%%%%%%%%%%%%%%%%%%%%
%%%%%%%%%%%%%%%%%%%%%%% Acknowledgments %%%%%%%%%%%%%%%%%%%%
%%%%%%%%%%%%%%%%%%%%%%%%%%%%%%%%%%%%%%%%%%%%%%%%%%%%%%%%%%%%
\preface
ขอบคุณอาจารย์ที่ปรึกษา กรรมการ พ่อแม่พี่น้อง และเพื่อนๆ คนที่ช่วยให้งานสำเร็จ ตามต้องการ

%%%%%%%%%%%%%%%%%%%%%%%%%%%%%%%%%%%%%%%%%%%%%%%%%%%%%%%%%%%%%
%%%%%%%%%%%%%%%% ToC, List of figures/tables %%%%%%%%%%%%%%%%
%%%%%%%%%%%%%%%%%%%%%%%%%%%%%%%%%%%%%%%%%%%%%%%%%%%%%%%%%%%%%
% The three commands below automatically generate the table 
% of content, list of tables and list of figures
\tableofcontents                    
\listoftables
\listoffigures                      

%%%%%%%%%%%%%%%%%%%%%%%%%%%%%%%%%%%%%%%%%%%%%%%%%%%%%%%%%%%%%%
%%%%%%%%%%%%%%%%%%%%% List of symbols page %%%%%%%%%%%%%%%%%%%
%%%%%%%%%%%%%%%%%%%%%%%%%%%%%%%%%%%%%%%%%%%%%%%%%%%%%%%%%%%%%%
% You have to add this manually..
\listofsymbols
\begin{flushleft}
\begin{tabular}{@{}p{0.07\textwidth}p{0.7\textwidth}p{0.1\textwidth}}
\textbf{SYMBOL}  & & \textbf{UNIT} \\[0.2cm]
% $\alpha$ & Test variable\hfill & m$^2$ \\
% $\lambda$ & Interarival rate\hfill &  jobs/second\\
% $\mu$ & Service rate\hfill & jobs/second\\
\end{tabular}
\end{flushleft}
%%%%%%%%%%%%%%%%%%%%%%%%%%%%%%%%%%%%%%%%%%%%%%%%%%%%%%%%%%%%%%
%%%%%%%%%%%%%%%%%%%%% List of vocabs & terms %%%%%%%%%%%%%%%%%
%%%%%%%%%%%%%%%%%%%%%%%%%%%%%%%%%%%%%%%%%%%%%%%%%%%%%%%%%%%%%%
% You also have to add this manually..
\listofvocab
\begin{flushleft}
\begin{tabular}{@{}p{1in}@{=\extracolsep{0.5in}}l}
  OCR &  Optical Character Recognition \\
\end{tabular}
\end{flushleft}

%\setlength{\parskip}{1.2mm}

%%%%%%%%%%%%%%%%%%%%%%%%%%%%%%%%%%%%%%%%%%%%%%%%%%%%%%%%%%%%%%%
%%%%%%%%%%%%%%%%%%%%%%%% Main body %%%%%%%%%%%%%%%%%%%%%%%%%%%%
%%%%%%%%%%%%%%%%%%%%%%%%%%%%%%%%%%%%%%%%%%%%%%%%%%%%%%%%%%%%%%%


\chapter{บทนำ}

\section{ที่มาและความสำคัญ}
\hspace*{1cm}ในปัจจุบัน ได้เกิด คดีต่างๆมากมาย จำนวนของ ทนายนั้นมีไม่เพียงพอ ในการ จัดการกับแต่ละคดี เลยจำเป็นที่ทนาย 1 คนจะต้องดูแล ในหลายคดีพร้อมๆ กัน ซึ่งในแต่ละคดีนั้น ทนายต้องเป็นคน จัดการกับเอกสารมากมาย ซึ่ง ทนายต้องจำให้ได้ว่า เอกสารไหน ที่ใช้กับคดีของคนไหน ซึ่งอาจจะ เกิดข้อผิด พลาด หรือเกิดความล่าช้า การการดำเนินการ ในแต่ละคดีได้ ดังนั้น จึงอยากทำระบบ DMS เพื่อให้ทนายนั้น สามารถ ทำงานได้มีประสิทธิภาพมากยิ่งขึ้น โดยมีประชาชนทั่วไปที่ต้องการจะติดตามรูปคดีของคดีต่างๆที่เกิดขึ้นนั้น สามารถเข้าถึงได้ยาก หรือไม่ก็ต้องการที่จะดูเอกสารของเคสต่างๆที่คนทั่วไปสามารถเข้าดูได้ แต่ว่าข้อมูลเหล่านี้เข้าถึงได้ยาก หรือบางคนก็ไม่รู้คำว่าจะต้องค้นหาอย่างไร \newline
\hspace*{1cm}ด้วยเหตุเหล่านี้ทางคณะผู้จัดทำได้เล็งเห็นถึงปัญหาดังกล่าว จึงต้องการที่จะพัฒนาเว็บแอปพลิเคชันเพื่อใช้ในการจัดการเอกสารต่างๆของทนาย โดยสามารถที่จะนำเอกสารบางส่วนที่สามารถเผยแพร่ได้ ให้ทุกคนสามารถเข้าถึงได้ และนำข้อมูลต่างๆมาแสดงผลในรูปแบบของแผนที่หรือแผนภูมิรูปต่างๆ

\section{วัตถุประสงค์}
\begin{enumerate}
  \item เพื่อศึกษาระบบในการบริหารจัดการเอกสาร การติดตาม และขั้นตอนการดำเนินคดีต่างๆ ของสำนักทนายความ
  \item เพื่อออกแบบฐานข้อมูล ให้เหมาะสมหรือสอดคล้องกับการค้นหาและจัดเก็บเอกสารอิเล็กทอนิค
  \item เพื่อพัฒนาเว็บแอปพลิเคชันที่ใช้ในการบริหารจัดการเอกสาร
  \item เพื่อสร้างช่องทางเข้าถึงข้อมูลเบื้องต้นอย่างง่ายให้กับทนายความและลูกความ
\end{enumerate}

\section{ขอบเขตและข้อจำกัดของปริญญานิพนธ์}
\hspace*{1cm}เว็บแอปพลิเคชันของคณะผู้จัดทำสามารถทำงานได้บนคอมพิวเตอร์ผ่านเว็บเบราว์เซอร์ต่าง ๆ เท่านั้นอาทิเช่น Google Chrome, Safari หรือ Microsoft Edge อีกทั้งข้อมูลที่นำมาใช้ในปริญญานิพนธ์นี้จะเป็นข้อมูลจริงที่ถูกเผยแพร่ผ่านเว็บไซต์ของศาลยุติธรรมซึ่งข้อมูลนี้ทุกคนสามารถเข้าถึงข้อมูลเหล่านี้ได้ และอีกส่วนหนึ่งเป็นข้อมูลเอกสารที่ได้ความยินยอมจากเจ้าของคดีและทนายให้สามารถเผยแพร่และเป็นตัวอย่างในปริญญานิพนธ์นี้เท่านั้น

\section{เนื้อหาทางวิศวกรรมที่เป็นต้นฉบับ}
\hspace*{1cm}โครงงานนี้พัฒนาขึ้นจากการใช้ความรู้ในด้านการออกแบบฐานข้อมูล ความรู้ในการพัฒนาเว็บแอปพลิเคชัน รวมไปถึงความรู้ทางด้านวิศวกรรมซอฟต์แวร์ ซึ่งไม่ว่าจะเป็นการออกแบบ UML diagram ต่างๆ การวางแผนการทำงาน การพัฒนาซอฟต์แวร์ในรูปแบบของ Waterfall การใช้ Kanbann เพื่อช่วยในการจัดการการทำงาน มาใช้เพื่อสร้างเป็นเว็ปแอปพลิเคชันสำหรับจัดการเอกสารของทนาย โดยมีการใช้ Elasticsearch ในการค้นหาเอกสาร และใช้ Gin framework ที่ใช้ภาษา Go language มาจัดการส่วนของการประมวณผล สุดท้ายได้นำ NextJs Framework ซึ่งใช้ภาษา Typescript , component framework และ CSS framework มาใช้ในการ แสดงข้อมูลที่ถูกประมวลผลแล้ว 

\newpage
\section{ขั้นตอนการทํางานและระยะเวลาการดําเนินงาน}
\begin{enumerate}
  \item เลือกหัวข้อที่สนใจ
  \item ศึกษาหัวข้อปริญญานิพนธ์ที่ได้รับ
  \item จัดทำข้อเสนอหัวข้อปริญญานิพนธ์
  \item ศึกษาทฤษฎี เทคโนโลยีและซอฟท์แวร์ที่ใช้
  \item เก็บโครงสร้างของข้อมูลที่ได้รับจากอุปกรณ์เครือข่าย
  \item วางโครงสร้างและรูปแบบของแอปพลิเคชั่น
  \item จัดทำรูปเล่มปริญญานิพนธ์ประจำภาคการศึกษาที่ 1/2565
  \item นำเสนอปริญญานิพนธ์ประจำภาคการศึกษาที่ 1/2565
  \item พัฒนาเว็บแอปพลิเคชัน
  \item ทดสอบเว็บแอปพลิเคชัน
  \item ปรับปรุงและแก้ไขเว็บแอปพลิเคชัน
  \item ทดสอบการใช้งานจริง
  \item จัดทำรูปเล่มปริญญานิพนธ์ประจำภาคการศึกษาที่ 2/2565
  \item นำเสนอปริญญานิพนธ์ประจำภาคการศึกษาที่ 2/2565
\end{enumerate}

\section{ประโยชน์ที่คาดว่าจะได้รับ}

โครงงานนี้จะเป็นประโยชน์กับใคร ยังไง ทั้งในเชิงรูปธรรมและนามธรรม ในปัจจุบันหรือในอนาคตถ้านำไป
ต่อยอด

\section{ตารางการดำเนินงาน}

\newpage
\section{ผลการดำเนินงาน}
ผลการดำเนินงานในภาคการศึกษาที่ 1/2565
\begin{enumerate}
  \item โครงสร้างและรูปแบบของเว็บแอปพลิเคชัน
  \item เก็บโครงสร้างของข้อมูลที่ได้รับจากอุปกรณ์เครือข่าย
  \item รูปเล่มปริญญานิพนธ์แสดงความคืบหน้าในภาคการศึกษาที่ 1/2565
\end{enumerate}
ผลการดำเนินงานในภาคการศึกษาที่ 2/2565
\begin{enumerate}
  \item เว็บแอปพลิเคชันที่ทำงานได้อย่างสมบูรณ์
  \item รูปเล่มปริญญานิพนธ์
\end{enumerate}

%%%%%%%%%%%%%%%%%%%%%%%%%%%%%%%%%%%%%%%%%%%%%%%%%%%%%%%%%%%%
%%%%%%%%%%%%%%%%%%%%%%% CHAPTER2 %%%%%%%%%%%%%%%%%%%%%%%%%%%
%%%%%%%%%%%%%%%%%%%%%%%%%%%%%%%%%%%%%%%%%%%%%%%%%%%%%%%%%%%%
\chapter{ทฤษฎีความรู้และงานที่เกี่ยวข้อง}

\hspace*{1cm}ในการจัดทําปริญญานิพนธ์เรื่อง Document management system for lawyer มีการอ้างอิงถึงทฤษฎี งานวิจัย และผลิตภัณฑ์ที่เกี่ยวข้องที่ใช้ในการพัฒนาชิ้นงาน ดังต่อไปนี้

\section{ทฤษฎีที่เกี่ยวข้อง}


\subsection{การดำเนินคดีของทนายความ} 
\hspace*{1cm}หลังจากที่ ทนายได้รับเรื่องคดีเข้ามาแล้ว ทนายก็จะทำการสอบข้อเท็จจริงจากลูกความ และทำการปรับข้อเท็จจริงให้เข้ากับหลักกฎหมาย จากนั้น จะทำการออกหนังสือกล่าวทวงถาม เขียนคำฟ้อง คำให้การ และคำให้การพร้อมฟ้องแย้ง หลังจากนั้นจะทำการยื่นคำฟ้อง หรือยื่นคำให้การให้แก่ศาล ต่อมาจะเป็นการ ยื่นบัญชีระบุพยาน และทำการขอหมายเรียก พยานเอกสาร พยานบุคคล หรือพยานวัตถุ จากนั้นจะเป็นการเตรียมคดี หรือซ้อมพยาน ก่อนที่จะนัดฟังคำพิพากษา ทั้งศาลชั้นต้น ศาลอุทธรณ์ และศาลฎีกา 

\subsection{Kanban software developments} 
\hspace*{1cm}Kanban board ถูกสร้างจากวิศวกรของบริษัท Toyota ที่มีชื่อว่า Taiichi Ohno ในปี 1940 โดนแนวคิดนี้สร้างขึ้นเพื่อควบคุมและจัดการงาน สินค้าคงคลังให้ในแต่ละขึ้นตอน ผลิตอย่างเหมาะสม และเป็นแบบแผนที่สุด\\
\hspace*{1cm}ดังนั้น Kanban board ถูกออกแบบมาเพื่อให้เห็นภาพรวม และช่วยควบคุม workflow ทั้งหมด ของโปรเจคนั้น ซึ่งจะมีลักษณะเป็น การดานหรือบอร์ด และการ์ด โดยในบอร์ดนั้น จะแสดงสเตตัสงานของทีม โดยจะมี post-it หรือการ์ด ที่จะมี Task หรืองานที่เราจะต้องทำ แปะอยู่ในสเตตัสนั้นๆ เพื่อทำให้เราสามารถรู้ได้ว่า เราควรจะต้องโฟกัสกำงานใดก่อน โดยสถานะพื้นฐานที่คนส่วนใหญ่ใช้กันก็จะมี To Do, Doing และDone แต่เราก็สามารถเพิ่มลดได้เองตามความเหมาะสมกับทีมและ เนื้องาน

\subsection{Database System}
\hspace*{1cm} Database หรือ ฐานข้อมูล คือ หมายถึง กลุ่มของข้อมูลที่ถูกเก็บรวบรวมไว้ โดยมีความสัมพันธ์กัน และถูกจัดเก็บอย่างมีระบบ เพื่อตอบสนองต่อความต้องการขององค์กร หรือระบบ โดยไม่จำเป็นต้องอยู่ในแฟ้มข้อมูลเดียวกัน สามารถแยกเก็บหลาย ๆ แฟ้มข้อมูลได้\\
\hspace*{1cm} Database System หรือ ระบบฐานข้อมูล หมายถึง การรวมกันของฐานข้อมูลที่เกี่ยวข้องกันตั้งแต่ 2 ฐานข้อมูลขึ้นไป โดยมีจุดประสงค์เพื่อลดความซ้ำซ้อนของข้อมูล และช่วยเพิ่มประสิทธิภาพในการจัดการข้อมูล ผ่านระบบจัดการฐานข้อมูล (Database Management System) ซึ่งการแบ่งชนิดของฐานข้อมูลสามารถแบ่งได้หลายหัวข้อ แต่ในที่นี้ผู้จัดทำจะแบ่งชนิดของฐานข้อมูลตามความนิยมใช้งาน โดยจะสามารถแบ่งออกได้เป็น 3 ประเภท ดังนี้
\begin{enumerate}
    \item {RDBMS (Relational Database Management System)}\\
    \hspace*{1cm}เป็นระบบจัดการฐานข้อมูลที่มีความเสถียรมากเหมาะสำหรับการเก็บข้อมูลที่มีจุดประสงค์และแยกประเภทชัดเจน ใช้ภาษา SQL ในการ Query และ Maintan Database มีการเก็บข้อมูลอยู่ในรูปแบบ ตาราง (Table) มีองค์ประกอบเป็น Rows และ Columns ซึ่งอำนวยความสะดวกในการเข้าถึงข้อมูลที่มีความสัมพันธ์กันระหว่างตาราง
    \item {NoSQL Database}\\
    \hspace*{1cm}NoSQL หรือ Non-relational database เป็นระบบจัดการฐานข้อมูลที่ไม่มีความสัมพันธ์กันอย่างชัดเจนหรือไม่มีรูปแบบที่แน่นอน เหมาะสำหรับการใช้งานกับ Big Data Real-time Web Application หรือ Logging โดยให้ความสำคัญกับเรื่องของความเร็วมากกว่าความถูกต้องแม่นยำ รวมถึงเหมาะกับข้อมูลชนิด Dynamic Data ที่มีการเปลี่ยนแปลงเกิดขึ้นบ่อยโดยไม่สนใจความสัมพันธ์ของข้อมูลก่อนหน้า
    \item {In-memory Database}\\
    \hspace*{1cm}เป็นระบบจัดการฐานข้อมูลชนิดหนึ่งที่ถูกสร้างขึ้นเพื่อวัตถุประสงค์ในการเก็บข้อมูลที่ต้องการเข้าถึงอย่างรวดเร็ว โดยใช้วิธีการเก็บ data ไว้ใน memory แทน ซึ่งมีความเร็วมากกว่า การจัดเก็บข้อมูลประเภทอื่น เช่น HDD อย่างไรก็ตาม In-memory database นั้นมีความเสี่ยงที่จะเกิดการสูญหายของข้อมูลได้มาก ดังนั้น In-memory database จึงเหมาะสำหรับ Application ที่ต้องการความรวดเร็วในการตอบสนองอย่างมาก
\end{enumerate}

\subsection{RESTful API}
\hspace*{1cm}RESTful API เป็นรูปแบบของสถาปัตยกรรมที่ใช้ Web Protocol ในการติดต่อสื่อสาร ระหว่างคอมพิวเตอร์สองเครื่อง ซึ่งในการติดต่อสื่อสาร แต่ละครั้งต้องมีการกำหนดรูปแบบการใช้งานก่อน เช่น GET POST เป็นต้น ซึ่งความสามารถ หลักๆของ RESTful นั้นคือสามารถส่งข้อมูล เก็บข้อมูล แสดงผล โดยจะมีทั้ง รูปภาพ, วิดีโอ, หรือเป็นข้อมูลที่เกี่ยวกับธุรกิจ\\
\hspace*{1cm}ซึ่งใน RESTful API จะเป็นการส่ง HTTP Request ไปยังเครื่องเซิฟเวอร์ โดยใน HTTP Request นั้นจะประกอบไปด้วย\\
\begin{itemize}
  \item {VERB คือเป็น HTTP Method เช่น GET POST PATCH DELETE เป็นต้น}
  \item {URI คือ จุดหมายหรือตำแหน่งที่ต้องการให้ระบบทำงาน}
  \item {HTTP Version คือเวอร์ชั่นของ HTTP ว่า HTTP Request นี้ใช้เวอร์ชั่นอะไร}
  \item {Request Header คือ metadata ที่เอาไว้ใช้เก็บ key-header เพื่อระบุข้อมูลของผู้ส่ง ผ่านทาง Header}
  \item {Request Body คือข้อมูล ในส่วนของ RESTful ที่เราจะทำการส่ง หรือได้รับ ซึ่งสามารถประกอบไปด้วย รูปภาพ วิดีโอ หรือข้อมูลที่เกี่ยวกับธุรกิจ}  
\end{itemize}

\subsection{Data visualization}
\hspace*{1cm} คือ การนำข้อมูลที่มีอยู่หรือจากแหล่งข้อมูลต่างๆ มาวิเคราะห์ประมวลผลแล้วนำเสนอออกมาในรูปแบบที่เข้าใจง่ายสามารถมองแล้วเข้าใจ เช่น รูปภาพ แผนที่ แผนภูมิประเภทต่างๆ ตาราง เป็นต้น ซึ่งประโยชน์ที่ได้จากการทำ Data visualization แล้วไม่ใช่เพียงแค่การอธิบายข้อมูลให้เข้าใจง่ายขึ้นแล้ว แต่ยังเป็นถึงการทำให้เห็นถึงภาพรวม หรือสามารถที่จะคาดการณ์ หรือเปรียบเทียบ และหาความสัมพันธ์ของข้อมูลได้ อีกทั้งยังประหยัดเวลาในการตีความข้อมูล หรือเห็นจุดเด่นของข้อมูลได้ชัดเจนยิ่งขึ้น 

\subsection{Optical Character Recognition}
\hspace*{1cm} Optical Character Recognition หรือ OCR คือเทคโนโลยีพื้นฐานในการเปลี่ยนข้อความที่อยู่ในเอกสารประเภทต่างๆให้อยู่ในรูปแบบของข้อความที่คอมพิวเตอร์สามารถเข้าใจได้ เพื่อใช้ในการเก็บข้อมูลเอกสารให้อยู่ในรูปแบบของตัวอักษรเพื่อใช้ในการค้นหา \\
\hspace*{1cm} ในปกติเราจะเห็นการใช้ OCR กับการ สแกนหนังสือเก่า สแกนเอกสาร การแปลงแบบฟอร์มกระดาษเป็นรูปแบบดิจิทัลเป็นต้น เพื่อที่จะนำไปเก็บรักษาในระบบดิจิทัล ซึ่ง OCR นี้จะช่วยประหยัดเวลาเป็นอย่างมาก อีกทั้งยังมีความแม่นยำ และยังสามารถ แปลงไฟล์ได้เป็นหลายประเภท เช่นแปลงจากรูปภาพ หรือไฟล์ PDF เป็นต้น





\section{งานวิจัยที่เกี่ยวข้องและผลิตภัณฑ์ใกล้เคียง}
\subsection{Avokaado application}
\hspace*{1cm} Avokaado คือ all-in-one contract lifecycle management ที่จะช่วยสำนักงานกฎหมาย ในการสร้าง จัดการ และดูแลเอกสารต่างๆ อยู่ในแพลตฟอร์มเดียว เช่น smart drafting, co-editing, negotiating approving, e-signing, document management และ tracking dashboard
% \begin{figure}[!h]\centering
    %   \includegraphics[width=13cm]{./image/NetScout.png}
    %   \caption{ รูปแสดงซอฟแวร์ nGeniusPLUSE ของ NETSCOUT }\label{fig:netScout}
    % \end{figure}
\subsection{ContractSafe application}
\hspace*{1cm} ContractSafe ใช้สำหรับจัดการกับเอกสารสัญญาต่างๆ โดยจะสามารถ ค้นหา และ แสกนเอกสารโดยใช้ OCR ได้ รวมไปถึงยังมีระบบความปลอดภัยที่ดีเยี่ยม \\
\begin{itemize}
  \item มีการจัดการจัดเก็บที่ปลอดภัย และสามารถแสกนได้อย่างรวดเร็ว	
  \item สามารถจัดการกับสิทธิการเข้าถึงข้อมูลได้และมีการติดตามที่ชัดเจน
\end{itemize}

\subsection{Laserfiche application}
\hspace*{1cm} Laserfiche สามารถปรับแต่งหรือกำหนดรูปแบบได้อย่าง อิสระ เพื่อให้สอดคล้องกับแต่ละบริษัท ทำให้ลดข้อผิดพลาดที่อาจจะเกิดขึ้น และสามารถดำเนินการได้อย่างครบถ้วน จุดเด่น ของLaserfiche นั้น จะเป็นรูปแบบ Config Flow คือ รูปแบบการทำงานที่จะสามารถย้อนกลับมาแก้ไขไฟล์ได้ตลอดเวลา ซึ่งถ้าหากเจอข้อผิดพลาด ก็ไม่ต้องยกเลิก หรือต้องเสียเวลาเริ่มต้นกระบวนการใหม่ทั้งหมด \\
\begin{itemize}
  \item ค้นหาเอกสารได้อย่างรวดเร็ว
  \item ค้นหาข้อมูลด้วย Text จากเอกสารสแกน
  \item จัดระเบียบข้อมูลและเอกสารได้ตามต้องการ
  \item ลดไฟล์เอกสารซ้ำซ้อน
\end{itemize}

\subsection{Amagno application}
\hspace*{1cm} Amagno คือ เครื่องมือสำคัญสำหรับองค์กรเพื่อบริหารจัดการคอนเทนท์ต่าง ๆ ที่มีอยู่มากมายและกระจัดกระจาย ให้กลับมาเป็นระเบียบ โดยจะสามารถตรวจสอบและค้นหาเอกสารได้อย่างรวดเร็ว อีกทั้งยังสามารถกำหนดสิทธิการเข้าถึงเพื่อเพิ่มความปลอดภัยให้กับข้อมูลที่สำคัญขององค์กร \\
\begin{itemize}
  \item ป้องกันเอกสารไม่ให้ถูกลบ - แก้ไขซ้ำ
  \item ป้องกันเอกสารซ้ำด้วยการเปรียบเทียบเวอร์ชั่นเอกสารล่าสุด
  \item สแกนเอกสารกระดาษเข้าระบบได้โดยตรง
  \item รองรับไฟล์ข้อมูลเอกสารได้ทุกรูปแบบ
\end{itemize}

\section{เทคนิคและเทคโนโลยีที่ใช้}
\subsection{Frontend framework}
\begin{itemize}
  \item \textbf{NextJS} \\
\hspace*{1cm} เป็น Framework ที่ใช้สำหรับเขียนเว็บ โดยพัฒนามาจาก React ซึ่ง ออกแบบมาให้ใช้งานง่ายมี learning curve ต่ำ โดยตัว NextJS มีคามโดดเด่นด้านการทำ SSR (Server-side-rendering) และยังมีการทำ Routing ที่เข้าใจได้ง่ายมี Folder structure ที่ชัดเจน \\
\hspace*{1cm} เนื่องจาก NextJS มี learning curve ที่ต่ำ และไม่จำเป็นต้อง optimize มากนัก เนื่องจากตัว Framework ได้ทำการ optimize มาในบางส่วนแล้ว เลยทำให้สามารถเริ่ม ทำงานได้เลย และอยู่ในมาตรฐานของเว็บไซต์ทั่วไป
  \item \textbf{Antd} \\
\hspace*{1cm} คือ React UI library ที่มี  Component ให้ใช้ที่ค่อนข้างครอบคลุม หลากหาย และสวยงาม ซึ่งสามารถรองรับ Responsive ได้เป็นอย่างดี \\
\hspace*{1cm} ดังนั้น เราจึงเลือกใช้ antd เพื่อที่จะมาลดเวลา การทำ Base component ต่างๆ ซึ่งคิดว่าในส่วนของการทำเว็บ ตัว Base component นั้นใช้เวลาในการ พัฒนามากที่สุด และ antd ยังสามารถ กำหนด Rule ต่างๆ เพื่อ handling error ต่างๆ ได้สะดวกขึ้น
  \item \textbf{Tailwind} \\
\hspace*{1cm} คือ Utility-First CSS Framework สำหรับจัดการ HTML Element โดยตรง ซึ่ง Tailwind ออกแบบมาให้เป็น low-level class ทำให้ตัวไฟล์มีขนาดที่เล็กมาก และจากที่ Tailwind มี class ให้ใช้ค่อนข้างครอบคลุม และการทำ Custom class element ได้ง่ายขึ้น และที่สำคัญ Tailwind เป็น CSS Framework ที่ทำ responsive ค่อนข้างง่าย \\
\hspace*{1cm} ดังนั้น การนำ Tailwind มาใช้ในการทำเว็บไซต์จะนำมาช่วยทำ layout ต่างๆในเว็บ และทำในส่วนของ custom class element ให้ง่าย และเร็วมากขึ้น
\end{itemize}
\subsection{Backend framework}
\begin{itemize}
  \item \textbf{Gin framework} \\
\hspace*{1cm} เป็น web Framework ที่เขียนด้วยภาษา Golang ซึ่งมีประสิทธิภาพด้านความเร็วที่สูงมาก และเป็นที่นิยมในกลุ่มนักพัฒนา ซึ่ง Gin นั้นสามารถที่จะสร้างเว็บแอปพลิเคชันและไมโครเซอร์วิสได้ ซึ่งในตัวของ framework ประกอบด้วยฟังก์ชันต่างๆให้ใช้งานครบถ้วน 
\end{itemize}
\subsection{Database}
\begin{itemize}
  \item \textbf{PostgreSQL} \\
\hspace*{1cm} เป็น RDBMS ที่ใช้งานอย่างแพร่หลาย และมีความสามารถหลากหลาย เช่น สามารถเก็บ Data type เป็น Array หรือ JSON ได้ อีกทั้งยังเป็น Open source ทำให้สามารถนำไปใช้งานได้ฟ
  \item \textbf{Elasticsearch} \\
\hspace*{1cm} เป็น database ซึ่งเปิดตัวในปี 2010 โดยบริษัท Elasticsearch N.V. ที่มีเครื่องมือค้นหาและการวิเคราะห์ที่ถูกสร้างขึ้นบน Apache Lucene ซึ่งถูกพัฒนาด้วยภาษา Java โดย Elasticsearch สามารถเก็บข้อมูลได้หลายประเภททั้งข้อความ ตัวเลข structured data และ unstructured data และสิ่งที่ทำให้เป็นจุดเด่นคือเรื่องของความเร็วในการจัดเก็บ ค้นหา และวิเคราะห์ข้อมูลในปริมาณมากๆได้ ซึ่งเป็นความเร็วในหน่วยมิลลิวินาที เนื่องจาก Elasticsearch มีการเก็บข้อมูลอยู่ในรูปแบบ JSON และทาง Elasticsearch จะทำการแปลงเป็น index และจัดเก็บไว้ ถ้าให้เปรียบเทียบก็คือการนำข้อมูลมาจัดเรียงให้มีลักษณะคล้ายกับสารบัญ เมื่อทำการค้นหาจะทำให้ค้นหาได้ไว อีกทั้ง Elasticsearch ยังมีระบบ relevancy เป็นการทำให้ระบบรู้ว่าควรจะแสดงผลลัพธ์อะไรให้ตรงกับที่ต้องการมากที่สุดซึ่งมีการใช้วิธีการประเมินที่เรียกว่า Term Frequency-Inverse Document Frequency (TF-IDF) ก็คือ ยิ่งข้อความหรือคำที่หานั้นตรงมีอยู่ในเอกสารเยอะขนาดไหน ก็แสดงว่าเอกสารหรือข้อมูลชุดนั้นก็มีความเกี่ยวข้องมากและยิ่งจำนวนเอกสารที่มีในการค้นหามากเท่าไหร่แสดงว่า คำหรือข้อความที่ค้นหานั้นมีความสำคัญน้อยลง ซึ่งการใช้วิธีการประเมินนี้ทำให้ได้ผลลัพธ์ที่ออกมาได้รวดเร็ว ซึ่งสามารถใช้งานได้โดยการใช้ Elasticsearch API ในการค้นหาได้ แต่ Elasticsearch ไม่ใช่เป็นเพียงแต่ database ที่มีเอาไว้ใช้ในการเก็บข้อมูลและใช้ในการค้นหาเท่านั้น แต่ว่ายังมีระบบแสดงผลที่เรียกว่า Kibana ซึ่งจะแสดงผลให้เห็นข้อมูลต่างๆให้อยู่ในรูปแบบแผนภูมิ แผนที่ และฟิลเตอร์ได้
\end{itemize}
\subsection{Version control}
\begin{itemize}
  \item \textbf{Github} \\
\hspace*{1cm} เนื่องจาก กลุ่มเรามีการใช้ auto CI/CD ซึ่ง github มีสิ่งนี้ให้อยู่แล้ว ประกอบกับในทีมค่อนข้างคุ้นเคยกับ github มากกว่า source version control ค่ายอื่นๆ
\end{itemize}

\subsection{Cloud}
\begin{itemize}
\item \textbf{Google cloud storage} \\
\hspace*{1cm} คือ บริการ Storage ของ Google cloud ที่ให้บริการฝากรูป, ไฟล์ ต่าง ๆ ประเภทข้อมูลที่เป็น Unstructure data ขนาดใหญ่ มีการเปิด API ในการจัดการไฟล์ต่าง ๆ ทำให้สามารถเขียนโปรแกรมเพื่ออัพโหลด หรือดาวโหลดไฟล์เพื่อนำมาใช้งานได้ \\
\hspace*{1cm} เนื่องจาก ผู้จัดทำมีการเก็บไฟล์ประเภทรูปและเอกสารต่าง ๆ เช่น รูปโปรไฟล์ผู้ใช้งาน เอกสารของทนาย จึงเลือกใช้ Google cloud storage ในเก็บข้อมูลเพื่อเรา ไม่ต้องจัดการกับการจัดเก็บไฟล์ อีกทั้งในอนาคตยังสามารถ scale ได้ง่ายเพราะว่าไฟล์มีการ จัดเก็บอยู่บน Cloud ไม่ใช่เครื่องใดเครื่องหนึ่ง
\item \textbf{Google cloud Virtual Machine} \\
\hspace*{1cm}	คือ บริการ Virtual Machine ของ Google cloud เพื่อใช้ในการ deploy ทั้งหลังบ้านและ database และอีกเหตุผลที่ใช้ vm ของ google เพราะ google cloud มี service ต่างๆอีกมากมายที่สามารถใช้งานได้ 
\item \textbf{AI GEN Optical Character Recognition} \\
\hspace*{1cm} คือ บริการเทคโนโลยี OCR หรือ Optical Character Recognition เป็นเทคโนโลยีพื้นฐานในการเปลี่ยนข้อความที่อยู่ในเอกสารประเภทต่างๆให้อยู่ในรูปแบบของข้อความที่คอมพิวเตอร์สามารถเข้าใจได้ เพื่อใช้ในการเก็บข้อมูลเอกสารให้อยู่ในรูปแบบของตัวอักษรเพื่อใช้ในการค้นหา ซึ่งเป็นบริการที่รองรับภาษาไทย \\
\hspace*{1cm} เนื่องจาก ในส่วนของข้อมูลต้นทางนั้น บางส่วนจะเป็นในรูปแบบเอกสาร ดังนั้นจึงมีความจำเป็นที่จะต้อง นำ OCR มาใช้ โดยจะเปลี่ยนจากรูปภาพเอกสารที่มีตัวอักษรให้เป็นเอกสารอิเล็กทรอนิค เพื่อทางผู้จัดทำ จะสามารถ นำเอกสารอิเล็กทอนิค ไปจัดเก็บไว้ใน ส่วนของ Google Cloud Storage ต่อไป 
\end{itemize}

\subsection{Planning \& Discuss tools}
\begin{itemize}
\item \textbf{Trello} \\
\hspace*{1cm} เป็นแอปพลิเคชันที่ใช้เพื่อจัดการวางแผนงานหรือขั้นตอนการทำงาน โดยจะทำเป็นลักษณะของ Kanban board ซึ่งสามารถติดตามสถานะต่างๆ และปริมาณงานได้
\item \textbf{Figma and FigJam} \\
\hspace*{1cm} เป็นแอปพลิเคชันที่มีความสามารถในการออกแบบ flow และ UI ของเว็บแอปพลิเคชันอีกทั้ง ยังสามารถใช้งานได้ง่ายผ่าน Web browser ได้โดยตรง ซึ่ง FigJam เป็นอีกฟังก์ชันหนึ่งในตัว Figma ที่มีลักษณะเป็นไวท์บอร์ดสามารถที่จะเขียนข้อความต่างๆได้ โดยเหมาะที่จะมาใช้ในการระดมความคิด
\item \textbf{Discord} \\
\hspace*{1cm} เป็นแอปพลิเคชันที่ใช้สำหรับการสื่อสารที่สามารถสื่อสารได้ทั้งเสียงและข้อความโดยสามารถแบ่งแยกห้องออกมาเป็นหมวดหมู่ได้
\end{itemize}



%%%%%%%%%%%%%%%%%%%%%%%%%%%%%%%%%%%%%%%%%%%%%%%%%%%%%%%%%%%%
%%%%%%%%%%%%%%%%%%%%%%% CHAPTER3 %%%%%%%%%%%%%%%%%%%%%%%%%%%
%%%%%%%%%%%%%%%%%%%%%%%%%%%%%%%%%%%%%%%%%%%%%%%%%%%%%%%%%%%%
\chapter{วิธีการดำเนินงาน กระบวนการและการออกแบบ}

ในส่วนบทที่ 3 จะกล่าวถึงรายโครงสร้างของระบบหรือรูปแบบการออกแบบต่างๆ อาทิเช่น UML diagram, Usecase diagram เป็นต้น

\section{ข้อกำหนดและความต้องการของระบบ}

\section{Database Design}

\section{UML Design}

\section{GUI Design}

\section{การออกแบบการทดลอง}
\subsection{ตัวชี้วัดและปัจจัยที่ศึกษา}
\subsection{รูปแบบการเก็บข้อมูล}




%%%%%%%%%%%%%%%%%%%%%%%%%%%%%%%%%%%%%%%%%%%%%%%%%%%%%%%%%%%%
%%%%%%%%%%%%%%%%%%%%%%% CHAPTER4 %%%%%%%%%%%%%%%%%%%%%%%%%%%
%%%%%%%%%%%%%%%%%%%%%%%%%%%%%%%%%%%%%%%%%%%%%%%%%%%%%%%%%%%%
\chapter{ผลการดำเนินงาน}

You can title this chapter as \textbf{Preliminary Results} ผลการดำเนินงานเบื้องต้น or \textbf{Work Progress} ความก้าวหน้าโครงงาน for the progress reports. Present implementation or experimental results here and discuss them.
ใส่เฉพาะหัวข้อที่เกี่ยวข้องกับงานที่ทำ 

\section{ประสิทฺธิภาพการทำงานของระบบ} 
\section{ความพึงพอใจการใช้งาน}
\section{การวิเคราะห์ข้อมูลและผลการทดลอง}

%%%%%%%%%%%%%%%%%%%%%%%%%%%%%%%%%%%%%%%%%%%%%%%%%%%%%%%%%%%%
%%%%%%%%%%%%%%%%%%%%%%% CHAPTER5 %%%%%%%%%%%%%%%%%%%%%%%%%%%
%%%%%%%%%%%%%%%%%%%%%%%%%%%%%%%%%%%%%%%%%%%%%%%%%%%%%%%%%%%%
\chapter{บทสรุป}

This chapter is optional for proposal and progress reports but 
is required for the final report.

\section{สรุปผลโครงงาน}
สรุปว่าโครงงานบรรลุตามวัตถุประสงค์ที่ตั้งไว้หรือไม่ อย่างไร 

\section{ปัญหาที่พบและการแก้ไข}
State your problems and how you fixed them.

\section{ข้อจำกัดและข้อเสนอแนะ}
ข้อจำกัดของโครงงาน What could be done in the future to make your projects better.

%%%%%%%%%%%%%%%%%%%%%%%%%%%%%%%%%%%%%%%%%%%%%%%%%%%%%%%%%%%%%%%
%%%%%%%%%%%%%%%%%%%% Bibliography %%%%%%%%%%%%%%%%%%%%%%%%%%%%%
%%%%%%%%%%%%%%%%%%%%%%%%%%%%%%%%%%%%%%%%%%%%%%%%%%%%%%%%%%%%%%%

%%%% Comment this in your report to show only references you have
%%%% cited. Otherwise, all the references below will be shown.
%\nocite{*}
%% Use the kmutt.bst for bibtex bibliography style 
%% You must have cpe.bib and string.bib in your current directory.
%% You may go to file .bbl to manually edit the bib items.

\makeatletter
\g@addto@macro{\UrlBreaks}{\UrlOrds}
\makeatother

\bibliographystyle{kmutt}
\bibliography{string,cpe}

%%%%%%%%%%%%%%%%%%%%%%%%%%%%%%%%%%%%%%%%%%%%%%%%%%%%%%%%%%%%%%%
%%%%%%%%%%%%%%%%%%%%%%%% Appendix %%%%%%%%%%%%%%%%%%%%%%%%%%%%%
%%%%%%%%%%%%%%%%%%%%%%%%%%%%%%%%%%%%%%%%%%%%%%%%%%%%%%%%%%%%%%%
\appendix{ชื่อภาคผนวกที่ 1}
\noindent{\large\bf ใส่หัวข้อตามความเหมาะสม} \\





\end{document}
